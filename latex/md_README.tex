Interfejs uzytkownika Wytlumacz do czego program sluzy Wybierz opcje co chcesz zrobic (menu) Wpisz dane ktore chcesz uzyc (tabela komorek) Co chcesz z nimi zrobic wybierz opcje (edytowanie komorek) Program po paru sekundach zwraca wynik operacji (edycja w tablicy i utworzenie nowej komorki z wynikiem) Zapyta czy chcesz powtrzozyc dana czynnosc( mozliwosc wykonania programu jeszcze raz) Jesli nie program, sie zamknie nie zapisujac danych trwale(za kazdym razem cala tablica bedzie resetowana)

Co program zrobi? Pobiera dane od uzytkownika ile wierszy ile kolumn Tworzy plik main.\+cpp Odczytaj wybor z menu (menu.\+cpp) i zrob to co nalezy zrobic Uzyc funkcje by latwiej korzystac z programu (wyswietl dane)(funkcja.\+h)(tablica\+\_\+wysw.\+cpp) Wypisz dane podane przez uzytkownika zapytaj czy sa poprawne Wyslij dane do programu (tablica.\+cpp) Wyslane dane uzyte w programie Przetworz dane w programie tak by uzytkownik byl zadowolony Zwroc wynik uzytkownikowi Zapytaj czy chce powtorzyc proces

\#\+Programowanie\+Obiektowe 